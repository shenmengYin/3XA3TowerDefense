\documentclass[12,english]{article}
\usepackage[letterpaper, portrait, margin=1in]{geometry}

\usepackage{amsmath}
\usepackage[T1]{fontenc}
\usepackage{babel}
\usepackage{textcomp}
\usepackage{titlesec}
\usepackage{hyperref}
\usepackage{xcolor}
\usepackage{booktabs}
\usepackage{placeins}
\usepackage{graphicx}

\graphicspath{ {img/} }

\hypersetup{
    bookmarks=true,         % show bookmarks bar?
      colorlinks=true,       % false: boxed links; true: colored links
    linkcolor=black,          % color of internal links (change box color with linkbordercolor)
    citecolor=green,        % color of links to bibliography
    filecolor=magenta,      % color of file links
    urlcolor=cyan           % color of external links
}

\begin{document}
\begin{titlepage}
    \begin{center}
        \vspace*{1cm}
        
        \Huge
        \textbf{Module Interface Specification}
        
        \vspace{0.5cm}
        \LARGE
        SFWRENG 3XA3
        
        \vspace{1.5cm}
        
\Large
        Group 30
		\\ Alan Yin (yins1)
		\\ Huajie Zhu (zhuh5)
		\\ Junni Pan (panj10)
        
        \vspace{1.5cm}
        
        \Large
        November 09, 2017
        
    \end{center}
\end{titlepage}

\newpage
\tableofcontents
\listoftables
\listoffigures

\newpage
\color{red}{
\section{Major Revision History}
\begin{table}[!htbp]
	\begin{tabular}{|l|l|}
	    \hline
		Date & Revision\\ \hline
		November 6, 2017 & Rough draft of sections\\ \hline
		November 7, 2017 & Revised sections \\ \hline
		November 9, 2017 & Revision 0 complete\\ \hline
		\color{red}{December 6, 2017} & \color{red}{Revision 1 complete}\\ \hline
	\end{tabular}
	\caption{Major Revision History}
\end{table}} \color{black}

\section{Module Hierarchy}
\begin{table}[!htbp]
        \begin{tabular}{ll}
        \toprule
        Level 1 & Level 2 \\
        \midrule
        Hardware Hiding Module & \\
        \midrule
        Behaviour Hiding Module 
        & MainWindow\\
        & AreaSelector\\
        & SelectAreaCaptureScreen\\
        & Data\\ 
		& Output\\
		\midrule
        Software Decision Module 
        &FrameCapture\\
        &Point2D\\
        \bottomrule
        \end{tabular}
        \caption{Module Hierarchy}
        \end{table}

\section{MIS of Game Module}
		\subsection{Interface Syntax}
		\subsubsection{Exported Access Programs}
		\begin{tabular}[pos]{|c|c|c|c|}
			
			\hline
			%	\label
			\textbf{Name}& \textbf{In} & \textbf{Out} & \textbf{Exceptions} \\ \hline
			main & args & - & SlickException\\ \hline
			
		\end{tabular}
		
		\subsection{Interface Semantics}
		\subsubsection{State Variables}
		Not Applicable
		
		\subsubsection{Environmental Variables}
		\color{red} external environment \color{black}
		
		\subsubsection{Assumptions}
		None
		
		\subsubsection{Access Program Semantics}
		main():
		
		Input: args
		
		Transition: Start the application, set the resolution to 1000*700
		
		Output: None
		
		Exceptions: SlickException
	
	
	
\section{MIS of GameController Module}
		\subsection{Interface Syntax}
		\subsubsection{Exported Access Programs}
		\begin{tabular}[pos]{|c|c|c|c|}
			
			\hline
			%	\label
			\textbf{Name}& \textbf{In} & \textbf{Out} & \textbf{Exceptions} \\ \hline
			GameController & TDMap & - & -\\ \hline
			setPanelAndButtonProperties & - & - & - \\ \hline
			setInitialValues & - & - & -\\ \hline
			setMainFrame & - & - & -\\ \hline
			startNewWave & - & - & -\\ \hline
			paintComponent & Graphics & - & -\\ \hline
			stateChanged & - & - & - \\  \hline
			setPlaybackSpeed & - & - & - \\ \hline
			doPause & - & - & - \\ \hline
			doReturnToMainMenu & - & - & - \\ \hline
			doStartWave & - & - & - \\ \hline
			doSelectTower & ActionEvent & - & - \\ \hline
			doUpgrade & - & - & - \\ \hline
			doSell & - & - & - \\ \hline
			doDisplayCritterInfo & - & - & - \\ \hline
			actionPerformed & ActionEvent & - & - \\ \hline
			Draw & - & - & - \\ \hline
			observerUpdate & - & - & - \\ \hline
			endGame & - & - & - \\ \hline
			disableAllGameButtons & - & - & - \\ \hline
			resetPlayerWaveStats & - & - & - \\ \hline
			spendMoney & int & - & - \\ \hline
			getControlPanel & - & GameControlPanel & - \\ \hline
			getPlayPanel & - & MapPanel & - \\ \hline
			updateTowerInfoText & - & - & - \\ \hline
			reactToLeftClick & - & - & - \\ \hline
			buildTower & Tower & - & - \\ \hline
			updateSelectedTowerInfoAndButtons & - & - & - \\ \hline
			reactToMouseMove & Point & - & - \\ \hline
			reactToRightClick & Point & - & - \\ \hline
			itemStateChanged & - & ItemEvent & - \\ \hline
		\end{tabular}
		
		\subsection{Interface Semantics}
		\subsubsection{State Variables}
		controlPanel: GameControlPanel - the game control panel\\
	    mainFrame: JFrame - the main frame of this game\\
	    bPause: JButton - a button to pause the game\\
	    bStartWave: JButton - abutton to start the wave\\
    	bUpgrade: JButton - a button to upgrade the selected tower\\
        bSell: JButton - a button to sell the selected tower\\
        jsSpeed: JSlides - a slide to chage the game speed\\
        cbStrategies: JComboBox<String> - a list of strategies\\
        bCritterInfo: JButton - a button to show the critter information\\
        timer: Timer - the timer\\
        gamePlayer: Player - the player\\
        waveStartMoney: int - the wave start money\\ 
	    waveStartLives: int - the wave start lives\\
        waveNumber: int - the wave number\\
        activeCritterIndex: int - the index of active critter\\
	    drawableEntities: ArrayList<DrawableEntity> - a list of drawable entities\\
        tdMap: TDMap - the map to use\\
	    crittersInWave: ArrayList<Critter> - a list of critter on map\\
	    towersOnMap: ArrayList<Tower> - a list of tower on map\\
        gamePaused: boolean - is the game paused\\
        gameOver: boolean - is the game over\\
        selectedTowerToBuild: String - name of a tower to build\\
        towerBeingPreviewed: Tower - a tower which is being previewed\\
        selectedTower: Tower - a tower which is selected\\
        selectedTile: Maptile - a maptile which is selected\\
	    artist: Artist\_Swing - a artist helper\\
        clock: GameClock - a clock helper\\
	    helpers: ArrayList<Helper> - a list of helpers\\
        subjects: ArrayList<Subject> - a list of subjects\\
		
	   
		\subsubsection{Environmental Variables}
		None
		\subsubsection{Assumptions}
		Variables should be set before trying to access them
		
		\subsubsection{Access Program Semantics}
		
		GameController():
		
		Input: none
		
		Transition: This takes a TDMap object as the map on which to play the game.
		
		Output: None
		
		Exceptions: None\\
		\\
		setPanelAndButtonProperties():
		
		Input: none
		
		Transition: set this to be our game panel, get all of our Swing objects, and add
		
		action listener
		
		Output: None
		
		Exception: none\\ 
		\\
		setInitialValues():
		
		Transition: sets the initial values of the variables for the game. Also
		
		initializes arrays and gets the instances of the singleton classes.
		
		Output: None
		
		Exception: None\\ 
		\\
		setMainFrame(mFrame):
		
		Input: JFrame
		
		Transition: sets the JFrame that the game is displayed on
		
		Output: None
		
		Exception: None\\
		\\
		startNewWave():
		
		Transition: start a new wave
		
		Output: Return Value that was accessed (yReleased)
		
		Exception: none\\
		\\
		paintComponent(g):
		
		Input: Graphics
		
		Transition: update and draw all drawableEntities.
		
		Output:
		
		Exception: none\\
		\\stateChanged(e):
		
		Input: ChangeEvent
		
		Transition: set the game speed 
		
		Output:
		
		Exception: none\\
		\\
		setPlaybackSpeed():
		
		Transition: This relates to how fast or slow the wave apparently appears to the
		
		player.
		
		Output: None
		
		Exception: none\\
		\\
		doPause():
		
		Transition: pause the game
		
		Output: None
		
		Exception: none\\
		\\
		doReturnToMainMenu():
		
		Transition: return ti the main menu
		
		Output: None
		
		Exception: none\\
		\\
		doStartWave():
		
		Transition: unpause and start the wave
		
		Output: None
		
		Exception: none\\
		\\
		doSelectTower():
		
		Transition: select the tower
		
		Output: None
		
		Exception: none\\
		\\
		doUpgrade():
		
		Transition: upgrade the power
		
		Output: None
		
		Exception: none\\
		\\
		doSell():
		
		Transition: sell the tower
		
		Output: None
		
		Exception: none\\
		\\
		doDisplayCritterInfo():
		
		Transition: display the critter information
		
		Output: None
		
		Exception: none\\
		\\
		doDisplayCritterInfo(arg0):
		
		Input: ActionEvent
		
		Transition: response to the acquired action
		
		Output: None
		
		Exception: none\\
		\\
		Draw():
		
		Transition: call the repaint method
		
		Output: None
		
		Exception: none\\
		\\
		observerUpdate():
		
		Transition: This will update the game whenever one of the subjects (Critters)of
		
		the Game Controller is changed. e.g. if a critter dies or a tower is upgraded.
		
		Output: None
		
		Exception: none\\
		\\
		endGame():
		
		Transition: Ends the game by disabling buttons, and pausing the clock.
		
		Output: None
		
		Exception: none\\
		\\
		disableAllGameButtons():
		
		Transition: disables all of the game buttons
		
		Output: None
		
		Exception: none\\
		\\
		resetPlayerWaveStats():
		
		Transition: resets the player's stats (so that a new game can be started with the
		
		same instance)
		
		Output: None
		
		Exception: none\\
		\\
		spendMoney():
		
		Transition: spends a certain amount of money of the Player
		
		Output: None
		
		Exception: none\\
		\\
		getControlPanel():
		
		Transition: return the controlPanel
		
		Output: GameControlPanel
		
		Exception: none\\
		\\
		getPlayPanel():
		
		Transition: return the gamePanel
		
		Output: MapPanel
		
		Exception: none\\
		\\
		updateInfoLabelText():
		
		Transition: updates the info text
		
		Output: None
		
		Exception: none\\
		\\
		updateTowerInfoText():
		
		Transition: updates the tower info text
		
		Output: None
		
		Exception: none\\
		\\
		reactToLeftClick(point):
		
		Input: Point
		
		Transition: This method selects an existing tower to upgrade it, or puts a new
		
		tower on the selected tile, of the desired type. Hence, react to left click
		
		Output: None
		
		Exception: none\\
		\\
		buildTower(t):
		
		Input: Tower
		
		Transition: builds a tower t and puts it in the drawable entities to be drawn.
		
		Output: None
		
		Exception: none\\
		\\
		updateSelectedTowerInfoAndButtons():
		
		Transition: enable upgrades if they have enough money and if the tower isn't at
		
		max level, and enable the sell button.
		
		Output: None
		
		Exception: none\\
		\\
		reactToMouseMove(point):
		
		Input: Point
		
		Transition: react to the mouse move
		
		Output: None
		
		Exception: none\\
		\\
		reactToRightClick(point):
		
		Input: Point
		
		Transition: a right click clears the current tower selection
		
		Output: None
		
		Exception: none\\
		\\
		itemStateChanged(e):
		
		Input: ItemEvent
		
		Transition: if our strategies combobox changed, we want to change the strategy of
		
		the selected tower
		
		Output: None
		
		Exception: none\\
		\\
		
		
	
	
	
\section{MIS of ArtistSwing Module}
	\subsection{Interface Syntax}
	    \subsubsection{Exported Constants}
	    PIXELWIDTH=1000\\
	    PIXELHEIGHT=700\\
	    GAMEPIXELHEIGHT = PIXELHEIGHT-100\\
		\subsubsection{Exported Access Programs}
		\begin{table}[!htbp]
		\begin{tabular}{|c|c|c|c|}
			\hline
			Name & In & Out & Exceptions \\ \hline
			setGridWidth & int & - & - \\ \hline
			setGridHeight & int & - & - \\ \hline
			drawEmptyCircle & Graphics,color,int &  - & - \\ \hline
			drawFilledCircle & Graphics,color,int &  - & - \\ \hline
			drawFilledQuad & Graphics,color,int &  - & - \\ \hline
			drawEmptyQuad & Graphics,color,int &  -& - \\ \hline
			drawMap & TDMap, Graphics &  - & - \\ \hline
			drawCritter & Critter,Graphics &  - & - \\ \hline
			drawTower & Tower,Graphics &  - & - \\ \hline
			drawShot & Tower,Critter,Graphics &  - & - \\ \hline
		\end{tabular}
	\end{table}
		
	\subsection{Interface Semantics}
		\subsubsection{State Variables}
		width: int\\
		height: int\\
		g: graphics\\
		c: color\\
		x: int\\
		y: int\\
		radius: int\\
		tdMap: TDMap\\
		crit: Critter\\
		tow: Tower\\
		
		
	

		\subsubsection{Access Program Semantics}
        setGridWidth(width):

		Input: int
		
		Transition: set gridwidth to width
		
		
		Exception - None\\
		\\
		setGridHeight(Height):
		
			Input: int
			
			Transition: set gridheight to height
			 
			Exception - none\\
			\\
		drawEmptyCircle(g, c, x, y, radius):
			
			Input: Graphics, Color, int, int, int
			 
			Transition - Sets the color, and draws a circle (oval with equal radii)
			  			 
			Exception - None\\ 
			 \\
		drawFilledCircle(g, c, x, y, radius):
		
		    Input: Graphics, Color, int, int, int
		     
		    Transition - sets the color, and draws a filled circle (oval with radii equal)
		     
		    Exception - None\\ 
			 \\
		drawFilledQuad(g, c, x, y, radius):
		
		    Input: Graphics, Color, int, int, int
		     
		    Transition - sets the color and draws the rectangle
		     
		    Exception - None\\ 
			 \\
        drawEmptyQuad(g, c, x, y, radius):
            
            Input: Graphics, Color, int, int, int
		    
		    Transition - sets the color and draws the empty rectangle
		    
		    Exception - None\\ 
			 \\
		drawMap(tdMap,g):
		    
		    Input: TDMap, Graphics
		    
		    Transition - draws the map, gets the width and the height, finds the width
		    
		    and height of each block, sets the thickness of the line to 1, goes through
		    
		    the map in a nested for loop, if we have a path tile, draw it brown, if we
		    
		    have a scenery tile, draw it green.
		    
		    Exception - None\\ 
			 \\
		drawCritter(crit, g):
		    
		    Input: Critter, Graphics
		       
		    Transition - gets the critter attribuets, drawing the space behind the
		    
		    critter, Drawing the actual critter, replace critter with image, replace
		    
		    critter with image, for Healthbar: we draw a green rectangle, and then draw a
		    
		    red rectangle of size depending on how damaged. we draw a green rectangle,
		    
		    and then draw a red rectangle of size depending on how damaged. supposing the
		    
		    critter is damaged , we draw the red part.
		    
		    Exception - None\\ 
			 \\
		drawTower(tow,g):
		    
		    Input: Tower, Graphics
		    
		    Transition - sets our stroke to be size 1, gets the tile width and height of
		    
		    the gamemap, our outline tower is either black or blue (blue if selected), we
		    
		    draw the tower's rectangular part, and the outline and then we draw the
		    
		    tower's circular part, replace tower with image. for upgrades, we draw a
		    
		    circle (in white) around the main circle of the tower for each upgrade level
		    
		    16 since max tower level is 4. (so the circle doesn't go out of bounds).
		    
		    Exception - None\\ 
			 \\
		drawShot:(tow,crit,g):
		    
		    Input: Tower, Critter, Graphics
		    
		    Transition - gets the tile width and height, get tower color info, we set the
		    
		    stroke to be thicker than usually (2) and we draw the line.
		    
		    Exception - None\\ 
			 \\
        
             

\section{MIS of CritterGenerator Module}
	\subsection{Interface Syntax}
		\subsubsection{Exported Access Programs}
		
	\begin{tabular}[pos]{|c|c|c|c|}
	\hline
	\textbf{Name}& \textbf{In} & \textbf{Out} & \textbf{Exceptions} \\ 
	\hline
	getGeneratedCritterWave & int, TDMap & ArrayList<Critter> & - \\ \hline
					
	\end{tabular}		
		
	\subsection{Interface Semantics}
		\subsubsection{State Variables}
		BASECRITTERS: int - the base amount of critters\\
		MAXWAVE: int - THE maximum wave number
		\subsubsection{Environmental Variables}
		None
		\subsubsection{Assumptions}
        None

		\subsubsection{Access Program Semantics}
		getGeneratedCritterWave(wavelevel, exampleMap):
		
		Input: int, TDMap
		
		Transition: generates a group of critters for a certain wave number. IF it is a 
		
		multiple of 5, we do a boss round, with boss (infinity) and grouped (shuriken) 
		
		critters.
		
		Exception: None
		\\

\section{MIS of Gameclock Module}
	\subsection{Interface Syntax}
		\subsubsection{Exported Access Programs}
		
	\begin{tabular}[pos]{|c|c|c|c|}
	\hline
	\textbf{Name}& \textbf{In} & \textbf{Out} & \textbf{Exceptions} \\ 
	\hline
	GameClock & - & - & -\\ \hline
	getInstance & - & GameClock & -\\ \hline
	deltaTime & - & int & -\\ \hline
	setDeltaTime & int & - & -\\ \hline
	pause & - & - & -\\ \hline
	unPause & - & - & -\\ \hline
					
	\end{tabular}		
		
	\subsection{Interface Semantics}
		\subsubsection{State Variables}
		dTime: int\\
		dt: int\\
		clock: GameClock\\
		

		\subsubsection{Access Program Semantics}
	GameClock():
		
		Input: None
		
		Transition: default tick is 1 second
		
		Exception: None\\
		\\
	getInstance():
		
		Input: None
		
		Transition: none
		
		Output: return the instance (OF WHICH THERE IS ONLY 1) of the clock
		
		Exception: \\
		\\
	deltaTime():

		Input: None
		
		Transition: None
		
		Output: return deltaTime
		
		Exception: none\\
		\\
	setDeltaTime(dt):
		
		Input: int
		
		Transition: set deltaTime to dt
		
		Exception: none\\
		\\
	pause():
		
		Input: none
		
		Transition: pause by setting deltaTime to 0
		
		
		Exception: none\\
		\\
	unPause():
		
		Input: none
		
		Transition: unpause by setting deltaTime to 1

		Exception: none\\
		\\

\section{MIS of MouseAndKeyboardHandler Module}
	\subsection{Interface Syntax}
		\subsubsection{Exported Access Programs}
		
	\begin{tabular}[pos]{|c|c|c|c|}
	\hline
	\textbf{Name}& \textbf{In} & \textbf{Out} & \textbf{Exceptions} \\ 
	\hline
	MouseAndKeyboardHandler & GameController & - & - \\ \hline
	mouseClicked & MouseEvent & - & - \\ \hline
	mousePressed & MouseEvent & - & - \\ \hline
	mouseReleased & MouseEvent & - & - \\ \hline
	mouseEntered & MouseEvent & - & - \\ \hline
	mouseExited & MouseEvent & - & - \\ \hline
	mouseMoved & MouseEvent & - & - \\ \hline
	mouseDragged & MouseEvent & - & - \\ \hline
					
	\end{tabular}		
		
	\subsection{Interface Semantics}
		\subsubsection{State Variables}
		gameController: GameController - the gameController that we are helping
		\subsubsection{Environmental Variables}
		\color{red} external environment \color{black}
		\subsubsection{Assumptions}
        None

		\subsubsection{Access Program Semantics}
		MouseAndKeyboardHandler(gameController):
		
		Input: GameController
		
		Transition: set the gameController to help
		
		Exception: None\\
		\\
		mouseClicked(event):
		
		Input: MouseEvent
		
		Transition: on mouse click, we alert the game controller
		
		Exception: None\\
		\\
		mousePressed(event):
		
		Input: MouseEvent
		
		Transition: on mouse prresed, we alert the game controller
		
		Exception: None\\
		\\
		mouseReleased(event):
		
		Input: MouseEvent
		
		Transition: on mouse released, we alert the game controller
		
		Exception: None\\
		\\
		mouseEntered(event):
		
		Input: MouseEvent
		
		Transition: on mouse entered, we alert the game controller
		
		Exception: None\\
		\\
		mouseExited(event):
		
		Input: MouseEvent
		
		Transition: on mouse exited, we alert the game controller
		
		Exception: None\\
		\\
		mouseMoved(event):
		
		Input: MouseEvent
		
		Transition: let the game controller know if the mouse is moved
		
		Exception: None\\
		\\
		mouseDragged(event):
		
		Input: MouseEvent
		
		Transition: let the game controller know if the mouse is dragged
		
		Exception: None
		\\
		
		
\section{MIS of Critter Module}
	\subsection{Interface Syntax}
	    \subsubsection{Exported Constants}
	    MAXCRITTERLEVEL = 50\\
	    MAXSPEED = 15\\
	    CRITTERMESSAGE = "Below is a description of each of the colored critters.\n\n" + "Yellow:\t\t\tBoss Critter. Very hard to kill\n\n" + "White:\t\t\tFast but weak\n\n" + "Red:\t\t\tSlightly below average\n\n" + "Pink:\t\t\tStrong but slow\n\n" + "Orange:\t\t\tCompletely resistant to fire and slow\n\n" + "Cyan:\t\t\tAverage Critter\n\n";  \\
		\subsubsection{Exported Access Programs}
		\begin{table}[!htbp]
		\begin{tabular}{|c|c|c|c|}
			\hline
			Name & In & Out & Exceptions \\ \hline
			critter & int,TDMap & - & - \\ \hline
			setInitialValues & - & - & - \\ \hline
			calculateLevelMultiplier & - &  - & - \\ \hline
			getIndexInPixelPath & - &  double & - \\ \hline
			getListPixelPath & - &  ArrayList<Point> & - \\ \hline
			setSlowFactor & double &  -& - \\ \hline
			getColor & - &  Color & - \\ \hline
			getPixelPosition & - & point & - \\ \hline
			hasReachedEnd & - & boolean & - \\ \hline
			isAlive & - &  boolean & - \\ \hline
			isBurning & - &  boolean & - \\ \hline
			getSize & - & int & - \\ \hline
			getLoot & - &  int & - \\ \hline
			getImage & - & image & - \\ \hline
			setHitboxRadius & int &  - & - \\ \hline
			getHitPoints & - &  double & - \\ \hline
			getMaxHitPoints & double &  - & - \\ \hline
			getRawSpeed & - &  double & - \\ \hline
			setRawSpeed & int &  - & - \\ \hline
			getLevel & - &  int & - \\ \hline
			setLevel & int &  - & - \\ \hline
			isActive & - &  boolean & - \\ \hline
			setActive & boolean &  - & - \\ \hline
			getSpeed & - &  double & - \\ \hline
			updateAndDraw & graphics &  - & - \\ \hline
			updateHealth & - &  - & - \\ \hline
			updatePositionAndDraq & graphics &  - & - \\ \hline
			moveAndDrawCritter & int,graphics &  - & - \\ \hline
			drawCritter & graphics &  - & - \\ \hline
			damage & double &  - & - \\ \hline
			slowCritter & int,double &  - & - \\ \hline
			damageOverTimeCritter& int,double &  - & - \\ \hline
		\end{tabular}
	\end{table}
		
	\subsection{Interface Semantics}
		\subsubsection{State Variables}
		currHitPoints: double\\
		maxHitPoints: double\\
		speed: double\\
		size: int\\
		regen: int\\
		resistance: double\\
		cColor: Color\\
		reward: int\\
		level: int\\
		name: string\\
		slowFactor: double\\
		slowTime: int\\
		image: Image\\
		beenSlowedFor: int\\
		damageOverTimeVal: double\\
		dotTime: int\\
		beenDOTFor: int\\
		burning: boolean\\
		pixelPosition: Point\\
		active: boolean\\
		alive: boolean\\
		reachedEnd: boolean\\
		pixelPathToFollow: ArrayList<Point>\\
		indexInPixelPath: double\\
		intIndexInPixelPath: int\\
        g: graphics\\
		
		
		
	

		\subsubsection{Access Program Semantics}
        critter(level, m):

		    Input: int, TDMap
		
		    Transition: set the level from input, sets the size to scale with the grid 
		    
		    size (bigger blocks = bigger critters), sets the initial values of the 
		    
		    critter attributes.
		
		
		    Exception - None\\
		\\
		setInitialValues():
		
			Input: none
			
			Transition: sets the initial values of the critter attributes.
			 
			Exception - none\\
			\\
		calculateLevelMultiplier():
			
			Input: none
			 
			Transition - calculates the current level multiplier of the critter,This will
			
			be called by extending critters, usually
			  			 
			Exception - None\\ 
			 \\
		getIndexInPixelPath():
		
		    Input: none
		     
		    Transition: none
		    
		    Output: return indexInPixelPath
		     
		    Exception - None\\ 
			 \\
		getListPixelPath():
		
		    Input: none
		     
		    Transition: none
		    
		    Output: return pixelPathToFollow
		     
		    Exception - None\\  
			 \\
        setSlowFactor(slowFactor):
            
            Input: double
		    
		    Transition - sets slow factor
		    
		    Exception - None\\ 
			 \\
		setDOTAmount(dot):
		    
		    Input: double
		    
		    Transition: set up amount of dot
		    
		    Exception - None\\ 
			 \\
		getColor():
		
		    Input: none
		     
		    Transition: none
		    
		    Output: return cColor
		     
		    Exception - None\\  
		    \\
		getPixelPosition():
		
		    Input: none
		     
		    Transition: none
		    
		    Output: return pixelPosition
		     
		    Exception - None\\  
		    \\
		hasReachedEnd():
		
		    Input: none
		     
		    Transition: none
		    
		    Output: return reachedEnd
		     
		    Exception - None\\  
		    \\
        isAlive():
		
		    Input: none
		     
		    Transition: none
		    
		    Output: return alive
		     
		    Exception - None\\  
		    \\
        isBurning():
		
		    Input: none
		     
		    Transition: none
		    
		    Output: return reachedEnd
		     
		    Exception - None\\  
		    \\
        getSize():
		
		    Input: none
		     
		    Transition: none
		    
		    Output: return size
		     
		    Exception - None\\  
		    \\
		getLoot():
		
		    Input: none
		     
		    Transition: none
		    
		    Output: return reward
		     
		    Exception - None\\  
		    \\
		getImage():
		
		    Input: none
		     
		    Transition: none
		    
		    Output: return image
		     
		    Exception - None\\  
		    \\
		setHitboxRadius(size):
		
		    Input:int
		     
		    Transition: set size
		   
		     
		    Exception - None\\  
		    \\
		getHitPoints():
		
		    Input: none
		     
		    Transition: none
		    
		    Output: return currHitPoints
		     
		    Exception - None\\  
		    \\
		getMaxHitPoints():
		
		    Input: none
		     
		    Transition: none
		    
		    Output: return maxHitPoints
		     
		    Exception - None\\  
		    \\
		getRawSpeed():
		
		    Input: none
		     
		    Transition: none
		    
		    Output: return speed
		     
		    Exception - None\\  
		    \\
		setRawSpeed(speed):
		
		    Input:int
		     
		    Transition: set speed
		   
		     
		    Exception - None\\  
		    \\
		getLevel():
		
		    Input: none
		     
		    Transition: none
		    
		    Output: return level
		     
		    Exception - None\\  
		    \\
	    setLevel(level):
		
		    Input:int
		     
		    Transition: set level
		   
		     
		    Exception - None\\  
		    \\
		isActive():
		
		    Input: none
		     
		    Transition: none
		    
		    Output: return active
		     
		    Exception - None\\  
		    \\
		    
		setActive(act):
		
		    Input:int
		     
		    Transition: set active
		   
		     
		    Exception - None\\  
		    \\  
		getSpeed():
		
		    Input: none
		     
		    Transition: none
		    
		    Output: return speed
		     
		    Exception - None\\  
		    \\
		updateAndDraw(g):
		
		    Input: graphics
		     
		    Transition: we only want to do something if the critter is active. See if we 
		    
		    are being slowed, if so, tick the total amount of time we have been slowed 
		    
		    for. See if we are being damaged over time, if so, tick the time we have been
		    
		    DOT for. update the health of the critter,update the position of the critter 
		    
		    and draw it.
		   
		     
		    Exception - None\\  
		    \\  
		   
		updateHealth():
		    
		    Input: none
		       
		    Transition - updates the health of the critter (called on every "tick" of the clock)
		    
		    Exception - None\\ 
			 \\
		updatePositionAndDraw(g):
		    
		    Input: Graphics
		    
		    Transition - updates the position (and draws it), called on every tick of clock
		    
		    Exception - None\\ 
			 \\
		moveAndDrawCritter(index,g):
		    
		    Input: int, Graphics
		    
		    Transition - Moves the critter to a given position and draws it as it moves.
		    
		    Exception - None\\ 
			 \\
	    moveAndDrawCritter(g):
		    
		   Input: Graphics
		    
		   Transition - draws the critter using the artist class
		    
		   Exception - None\\ 
			 \\
	    damage(dam):
		    
		   Input: double
		    
		   Transition - Damages the critter for a certain amount
		    
		   Exception - None\\ 
			 \\
		slowCritters(Factor, sTime):
		    
		   Input: double, int
		    
		   Transition - set the slow factor and slow time
		    
		   Exception - None\\ 
			 \\
	    damageOverTimeCritter(Factor, sTime):
		    
		   Input: double, int
		    
		   Transition - set the damage over time factor and time
		    
		   Exception - None\\ 
			 \\
        

\section{MIS of Tower Module}
	\subsection{Interface Syntax}
		\subsubsection{Exported Access Programs}
		
	\begin{tabular}[pos]{|c|c|c|c|}
	\hline
	\textbf{Name}& \textbf{In} & \textbf{Out} & \textbf{Exceptions} \\ 
	\hline
	Tower & String, Point, ArrayList<Critter> & - & - \\ \hline
	getSellPrice & - & int & - \\ \hline
	getUpPrice & - & int & - \\ \hline
	setStrategy & IStrategy & - & - \\ \hline
	getPosX & - & int & - \\ \hline
	getPosY & - & int & - \\ \hline
	getRange & - & int & - \\ \hline
	getName & - & String & - \\ \hline
	getImage & - & Image & - \\ \hline
	getEnabled & - & boolean & - \\ \hline
	setEnabled & boolean & - & - \\ \hline
	getColor & - & Color & - \\ \hline
	isSelected & - & boolean & - \\ \hline
	getStrategy & - & IStrategy & - \\ \hline
	setSelected & boolean & - & - \\ \hline
	getDefaultStrategy & - & String & - \\ \hline
	setColor & Color & - & - \\ \hline
	shootTarget & Criiter, Graphics & - & - \\ \hline
	upgradeTower & - & - & - \\ \hline
	
	\end{tabular}		
		
	\subsection{Interface Semantics}
		\subsubsection{State Variables}
		MAXTOWERLEVEL: int - the max level of a tower(4)\\
		DEFAULTSTRATEGY: String - the default strategy to use("Closest")\\
		position: Point - the position of the tower\\
		damage: double - the damage of the tower\\
    	rateOfFire: int - the fire rate of a tower\\
	    range: int - the fire range of the tower\\
	    sellPrice: int - the sell price of the tower\\	
	    upCost: int - the upgrade cost of the tower\\
	    name: String - name of the strategy\\
	    level: int - level of the tower\\
	    tColor: Color - color if the tower\\
	    shotColor: Color - shoot color of the tower\\
	    image: Image - model image of the tower\\
        icon: ImageIcon - the icon of the image\\
        strategy: IStrategy - the strategy to use\\ 
        enabled: boolean - if it is enabled\\
        selected: boolean - if it is selected\\
		\subsubsection{Environmental Variables}
		\color{red} external environment \color{black}
		\subsubsection{Assumptions}
        None

		\subsubsection{Access Program Semantics}
		Tower(n, p, crittersOnMap):
		
		Input: String, Point, ArrayList<Critter>
		
		Transition: constructor to construct a tower
		
		Exception: None\\
		\\
		getSellPrice():
		
		Transition: return the sell price of the tower
		
		Output: int
		
		Exception: None\\
		\\
		getUpPrice(event):
		
		Transition: return the upgrade price of the tower
		
		Output: int
		
		Exception: None\\
		\\
		setStrategy(strategy):
		
		Input: IStrategy
		
		Transition: set the strategy of the tower
		
		Exception: None\\
		\\
		getPosX():
		
		Transition: return the x position of the tower
		
		Output: int
		
		Exception: None\\
		\\
		getPosY():
		
		Transition: return the y position of the tower
		
		Output: int
		
		Exception: None\\
		\\
		getRange():
		
		Transition: return the attack range of the tower
		
		Output: int
		
		Exception: None\\
		\\
		getName():
		
		Transition: return the name of the tower
		
		Output: String
		
		Exception: None\\
		\\
		getImage():
		
		Transition: return the image of the tower
		
		Output: Image
		
		Exception: None\\
		\\
		getEnabled():
		
		Transition: return if it is enabled
		
		Output: boolean
		
		Exception: None\\
		\\
		setEnabled(state):
		
		Input: boolean
		
		Transition: set enabled or dis-enabled
		
		Exception: None\\
		\\
		getColor():
		
		Transition: return the color of the tower
		
		Output: Color
		
		Exception: None\\
		\\
		isSelected():
		
		Transition: return if it is selected
		
		Output: boolean
		
		Exception: None\\
		\\
		getStrategy():
		
		Transition: return the strategy of the tower
		
		Output: IStrategy
		
		Exception: None\\
		\\
		setSelected(s):
		
		Input: boolean
		
		Transition: set it is been selected or not 
		
		Exception: None\\
		\\
		getDefaultStrategy():
		
		Transition: return the name of default strategy of the tower
		
		Output: String
		
		Exception: None\\
		\\
		setColor(newColor):
		
		Input: Color
		
		Transition: set the color of the tower
		
		Exception: None\\
		\\
		shootTarget(target, g):
		
		Input: Criiter, Graphics
		
		Transition: deals damage to the criiter based on amount of damage of the tower
		
		Exception: None\\
		\\
		upgradeTower():
		
		Transition: upgrades the tower based on properties
		
		Exception: None\\
		\\
		

\section{MIS of TDMap Module}
	\subsection{Interface Syntax}
		\subsubsection{Exported Access Programs}
		
	\begin{tabular}[pos]{|c|c|c|c|}
	\hline
	\textbf{Name}& \textbf{In} & \textbf{Out} & \textbf{Exceptions} \\ 
	\hline
	TDMap & - & - & - \\ \hline
	TDMap & String & - & - \\ \hline
	initializeGrid & - & - \\ \hline
	getPIXELWIDTH & - & int & - \\ \hline
	getPIXELHEIGHT & - & int & - \\ \hline
	getGridWidth & - & int & - \\ \hline
	getGridHeight & - & int & - \\ \hline
	getPointsOfShortestPath & - & ArrayList<Point> & - \\ \hline
	updateAndDraw & Graphics & - & - \\ \hline
	print & - & - & - \\ \hline
	
					
	\end{tabular}		
		
	\subsection{Interface Semantics}
		\subsubsection{State Variables}
		PIXELWIDTH: int - the pixelwidth\\
		PIXELHEIGHT: int - the pixelheight \\
		gridWidth: int - the gridthwidth\\
		gridHeight: int - the gridthheight\\
		shortestPath: LinkedList<Integer> - the shortest path of the map ;
		\subsubsection{Environmental Variables}
		None
		\subsubsection{Assumptions}
        None

		\subsubsection{Access Program Semantics}
		TDMap():
		
		Transition: set route and wall image, the grid width and height as default
		
		Exception: None\\
		\\
		TDMap(add):
		
		Input: String
		
		Transition: load grass and wall of the map
		
		Exception: None\\
		\\
		initializeGrid():

		Transition: initializes the gridTile array to be all new MapTile objects
		
		Exception: None\\
		\\
		getPIXELWIDTH():

		Transition: return the PIXELWIDTH of the map
		
		Output: int
		
		Exception: None\\
		\\
		getPIXELHEIGHT():

		Transition: return the PIXELHEIGHT of the map
		
		Output: int
		
		Exception: None\\
		\\
		getGridWidth():

		Transition: return the GridWidth of the map
		
		Output: int
		
		Exception: None\\
		\\
		getGridHeight():

		Transition: return the GridHeight of the map
		
		Output: int
		
		Exception: None\\
		\\
		getPointsOfShortestPath():

		Transition: return the shortest path of the map
		
		Output: ArrayList<Point>
		
		Exception: None\\
		\\
		updateAndDraw():

		Transition: uses the artist to draw the map
		
		Exception: None\\
		\\
		print():
		
		Transition: This method provides an easy way to print out the grid to display the
		
		map. It also prints out the shortest path the critters will take to move from the
		
		Start cell to the End Cell.
		
		Exception: None

\section{MIS of Point Module}
		\subsection{Interface Syntax}
			\subsubsection{Exported Access Programs}
				\begin{tabular}[pos]{|c|c|c|c|}
					
					\hline
					%	\label
					\textbf{Name}& \textbf{In} & \textbf{Out} & \textbf{Exceptions} \\ \hline
					Point & integer, integer & - & -\\ \hline
					getX & - & int & -\\ \hline
					getY & - & int & -\\ \hline
					setX & int & - & -\\ \hline
					setY & int & - & -\\ \hline
					setPoint & int, int & - & - \\ \hline
					equals & Point & boolean & - \\ \hline
					
				\end{tabular}
				
		\subsection{Interface Semantics}
			\subsubsection{State Variables}
			X: int - x value of point\\
			Y: int - y value of point
			
			\subsubsection{Environmental Variables}
			Not Applicable
			
			\subsubsection{Assumptions}
			Variables should be set before trying to access them
			
			\subsubsection{Access Program Semantics}
			setX(x):
			
			Input: Integer of X value
			
			Transition: updates the X value of the point
			
			Output: none
			
			Exception: none\\
			\\
			setY(y):
			Input: Integer of Y value
			
			Transition: updates the Y value of the point
			
			Output: none
			Exception: none\\
			\\
			getX():
			
			Input: none
			
			Transition: accesses the X value
			
			Output: Returns the X value of the Point
			
			Exception: none\\
			\\
			getY():
			
			Input: none
			
			Transition: accesses the Y value
			
			Output: Returns the Y value of the point
			
			Exception: none\\
			\\
			setPoint(x, y):
			
			Input: int, int
			
			Transition: set both coords of a point at once

			Output: none
			
			Exception: none\\
			\\
			equals(p):
			
			Input: Point
			
			Transition: check if one point equals another

			Output: boolean
			
			Exception: none\\
			\\

\section{MIS of Player Module}
		\subsection{Interface Syntax}
			\subsubsection{Exported Access Programs}
				\begin{tabular}[pos]{|c|c|c|c|}
					
					\hline
					%	\label
					\textbf{Name}& \textbf{In} & \textbf{Out} & \textbf{Exceptions} \\ \hline
					Player & - & - & -\\ \hline
					getInstance & - & Player & -\\ \hline
					getLives & - & int & -\\ \hline
					getMoney & - & int & -\\ \hline
					setLives & int & - & -\\ \hline
					setMoney & int & - & - \\ \hline
					addToMoney & int & - & - \\ \hline
					takeAwayALife & - & - & - \\ \hline
					getStartingLives & - & int & - \\ \hline
					getStartingMoney & - & int & - \\ \hline
					resetStats & - & - & - \\ \hline
					
				\end{tabular}
				
		\subsection{Interface Semantics}
			\subsubsection{State Variables}
			STARTINGLIVES: int - the default starting live
			STARTINGMONEY: int - the default starting money
			lives: int - the current lives
			money: int - the current money
			playerInstance: Player - the player
			
			\subsubsection{Environmental Variables}
			\color{red} external environment \color{black}
			
			\subsubsection{Assumptions}
			Variables should be set before trying to access them
			
			\subsubsection{Access Program Semantics}
			Player():
			
			Transition: The constructor
			
			Output: none
			
			Exception: none\\
			\\
			getInstance():
			
			Transition: return the playerInstance
			
			Output: Player
			
			Exception: none\\
			\\
			getLives():
			
			Transition: return the current lives
			
			Output: int
			
			Exception: none\\
			\\
			getMoney():
			
			Transition: return the current money
			
			Output: int
			
			Exception: none\\
			\\
			setLives(l):
			
			Input: int
			
			Transition: set the value of lives
			
			Exception: none\\
			\\
		    setMoney(m):
			
			Input: int
			
			Transition: set the value of money
			
			Exception: none\\
			\\
			addToMoney(moneyToAdd):
			
			Input: int
			
			Transition: add money to the player
			
			Exception: none\\
			\\
			takeAwayALife():
			
			Transition: reduce the player's lives by one
			
			Exception: none\\
			\\
			getStartingLives():
			
			Transition: return the default starting lives
			
			Output: int
			
			Exception: none\\
			\\
			getStartingMoney():
			
			Transition: return the default starting money
			
			Output: Player
			
			Exception: none\\
			\\
			resetStats():
			
			Transition: reset the stats of the player
			
			Output: Player
			
			Exception: none\\
			\\
		

\section{MIS of Closest Module}
	\subsection{Interface Syntax}
		\subsubsection{Exported Access Programs}
		
	\begin{tabular}[pos]{|c|c|c|c|}
	\hline
	\textbf{Name}& \textbf{In} & \textbf{Out} & \textbf{Exceptions} \\ 
	\hline
	Critter findTarget & Tower, ArrayList<Critter> & Critter & - \\ \hline
	toString & - & string & - \\ \hline
					
	\end{tabular}		
		
	\subsection{Interface Semantics}
		\subsubsection{State Variables}
		tower: Tower\\
	    g1: ArrayList<Critter>\\
		\subsubsection{Environmental Variables}
		None
		\subsubsection{Assumptions}
        None

		\subsubsection{Access Program Semantics}
		findTarget(tower, g1):
		
		Input: Tower, ArrayList<Critter>
		
		Transition: finds the target based on who is closest, set arbitrary large number 
		
		that will never be reached
		
		Output: return closest enemy
		
		Exception: None\\
		\\
		toString():
		
		Input: none
		
		Transition: none
		
		Output: return closest 
		
		Exception: None\\
		\\

\section{MIS of Farthest Module}
	\subsection{Interface Syntax}
		\subsubsection{Exported Access Programs}
		
	\begin{tabular}[pos]{|c|c|c|c|}
	\hline
	\textbf{Name}& \textbf{In} & \textbf{Out} & \textbf{Exceptions} \\ 
	\hline
	Critter findTarget & Tower, ArrayList<Critter> & Critter & - \\ \hline
	toString & - & string & - \\ \hline
					
	\end{tabular}		
		
	\subsection{Interface Semantics}
		\subsubsection{State Variables}
		tower: Tower\\
	    g1: ArrayList<Critter>\\
		\subsubsection{Environmental Variables}
		None
		\subsubsection{Assumptions}
        None

		\subsubsection{Access Program Semantics}
		findTarget(tower, g1):
		
		Input: Tower, ArrayList<Critter>
		
		Transition: finds the Critter that is farthest along the path
		
		Output: return farthest enemy
		
		Exception: None\\
		\\
		toString():
		
		Input: none
		
		Transition: none
		
		Output: return Farthest 
		
		Exception: None\\
		\\      
\section{MIS of Fastest Module}
	\subsection{Interface Syntax}
		\subsubsection{Exported Access Programs}
		
	\begin{tabular}[pos]{|c|c|c|c|}
	\hline
	\textbf{Name}& \textbf{In} & \textbf{Out} & \textbf{Exceptions} \\ 
	\hline
	Critter findTarget & Tower, ArrayList<Critter> & Critter & - \\ \hline
	toString & - & string & - \\ \hline
					
	\end{tabular}		
		
	\subsection{Interface Semantics}
		\subsubsection{State Variables}
		tower: Tower\\
	    g1: ArrayList<Critter>\\
		\subsubsection{Environmental Variables}
		None
		\subsubsection{Assumptions}
        None

		\subsubsection{Access Program Semantics}
		findTarget(tower, g1):
		
		Input: Tower, ArrayList<Critter>
		
		Transition: finds target that is fastest
		
		Output: return fastest enemy
		
		Exception: None\\
		\\
		toString():
		
		Input: none
		
		Transition: none
		
		Output: return Fastest
		
		Exception: None\\
		\\
\section{MIS of Strongest Module}
	\subsection{Interface Syntax}
		\subsubsection{Exported Access Programs}
		
	\begin{tabular}[pos]{|c|c|c|c|}
	\hline
	\textbf{Name}& \textbf{In} & \textbf{Out} & \textbf{Exceptions} \\ 
	\hline
	Critter findTarget & Tower, ArrayList<Critter> & Critter & - \\ \hline
	toString & - & string & - \\ \hline
					
	\end{tabular}		
		
	\subsection{Interface Semantics}
		\subsubsection{State Variables}
		tower: Tower\\
	    g1: ArrayList<Critter>\\
		\subsubsection{Environmental Variables}
		\color{red} external environment \color{black}
		\subsubsection{Assumptions}
        None

		\subsubsection{Access Program Semantics}
		findTarget(tower, g1):
		
		Input: Tower, ArrayList<Critter>
		
		Transition: finds the strongest enemy
		
		Output: return Strongest enemy
		
		Exception: None\\
		\\
		toString():
		
		Input: none
		
		Transition: none
		
		Output: return Strongest
		
		Exception: None\\
		\\
\section{MIS of Weakest Module}
	\subsection{Interface Syntax}
		\subsubsection{Exported Access Programs}
		
	\begin{tabular}[pos]{|c|c|c|c|}
	\hline
	\textbf{Name}& \textbf{In} & \textbf{Out} & \textbf{Exceptions} \\ 
	\hline
	Critter findTarget & Tower, ArrayList<Critter> & Critter & - \\ \hline
	toString & - & string & - \\ \hline
					
	\end{tabular}		
		
	\subsection{Interface Semantics}
		\subsubsection{State Variables}
		tower: Tower\\
	    g1: ArrayList<Critter>\\
		\subsubsection{Environmental Variables}
		None
		\subsubsection{Assumptions}
        None

		\subsubsection{Access Program Semantics}
		findTarget(tower, g1):
		
		Input: Tower, ArrayList<Critter>
		
		Transition: finds the Weakest enemy
		
		Output: return Weakest enemy
		
		Exception: None\\
		\\
		toString():
		
		Input: none
		
		Transition: none
		
		Output: return Weakest
		
		Exception: None\\
		\\

		
\section{MIS of GameApplicationFrame Module}
	\subsection{Interface Syntax}
	\subsubsection{Exported Constants}
	    PIXELWIDTH=ArtistSwing.PIXELWIDTH\\
	    PIXELHEIGHT=ArtistSwing.PIXELHEIGHT\\
	    APPNAME = "Group 30 Tower Defense"\\
	    TIMEOUT = 30\\
		\subsubsection{Exported Access Programs}
		
	\begin{tabular}[pos]{|c|c|c|c|}
	\hline
	\textbf{Name}& \textbf{In} & \textbf{Out} & \textbf{Exceptions} \\ 
	\hline
	GameApplicationFrame & TDMap & - & - \\ \hline
	init & - & - & - \\ \hline
	
					
	\end{tabular}		
		
	\subsection{Interface Semantics}
		\subsubsection{State Variables}
		controlPanel: GameControlPanel\\
	    mapPanel: MapPanel\\
	    gameController: GameController\\
	    tdMap: TDMap\\
		\subsubsection{Environmental Variables}
		None
		\subsubsection{Assumptions}
        None

		\subsubsection{Access Program Semantics}
		GameApplicationFrame(tdMap):
		
		Input: TDMap
		
		Transition: set the tdmap
		
		Exception: None\\
		\\
		init():
		
		Input: none
		
		Transition: set the Frame properties, get the control and map panels, add them to
		
		the frame, set the x button as the default close operation
		
		Exception: None\\
		\\	
\section{MIS of GameState Module}
	\subsection{Interface Syntax}
	
		\subsubsection{Exported Access Programs}
		
	\begin{tabular}[pos]{|c|c|c|c|}
	\hline
	\textbf{Name}& \textbf{In} & \textbf{Out} & \textbf{Exceptions} \\ 
	\hline
	init & GameContainer, StateBasedGame & - & - \\ \hline
	render & GameContainer, StateBasedGame, Graphics & - & - \\ \hline
	update & GameContainer, StateBasedGame, init & - & - \\ \hline
	getID & - & int & - \\ \hline
	
	
					
	\end{tabular}		
		
	\subsection{Interface Semantics}
		\subsubsection{State Variables}
		mapToLoad: TDMap\\
	    arg0: GameContainer\\
	    arg1: StateBasedGame\\
	    g: Graphics\\
	    delta: init\\
		\subsubsection{Environmental Variables}
		\color{red} external environment \color{black}
		\subsubsection{Assumptions}
        None

		\subsubsection{Access Program Semantics}
		init(arg0, arg1):
		
		Input: GameContainer, StateBasedGame
		
		Transition: \color{red} set main menu image, set background image and set background music \color{black}
		
		Exception: None\\
		\\
		render(arg0, arg1, g):
		
		Input: GameContainer, StateBasedGame, Graphics
		
		Transition: draw string at interface
		
		Exception: None\\
		\\	
		update(arg0, arg1, delta):
		
		Input: GameContainer, StateBasedGame, init
		
		Transition: set default map
		
		Exception: None\\
		\\
		getID():
		
		Input: none
		
		Transition: none
		
		Output: return 1
		
		Exception: None\\
		\\
\section{MIS of MainMenu Module}
	\subsection{Interface Syntax}
	
		\subsubsection{Exported Access Programs}
		
	\begin{tabular}[pos]{|c|c|c|c|}
	\hline
	\textbf{Name}& \textbf{In} & \textbf{Out} & \textbf{Exceptions} \\ 
	\hline
	init & GameContainer, StateBasedGame & - & - \\ \hline
	render & GameContainer, StateBasedGame, Graphics & - & - \\ \hline
	update & GameContainer, StateBasedGame, init & - & - \\ \hline
	getID & - & int & - \\ \hline
	
	
					
	\end{tabular}		
		
	\subsection{Interface Semantics}
		\subsubsection{State Variables}
		mapToLoad: TDMap\\
	    container: GameContainer\\
	    arg1: StateBasedGame\\
	    g: Graphics\\
	    delta: init\\
	    MainMenu: image\\
	    playNow: image\\
	    exitGame: image\\
	    mapToLoad: image\\
	    
		\subsubsection{Environmental Variables}
		None
		\subsubsection{Assumptions}
        None

		\subsubsection{Access Program Semantics}
		init(container, arg1):
		
		Input: GameContainer, StateBasedGame
		
		Transition: set image to mainMenu, set image to playNow, set image to exitGame
		
		Exception: None\\
		\\
		render(arg0, arg1, g):
		
		Input: GameContainer, StateBasedGame, Graphics
		
		Transition: draw string at interface, draw mainMenu image at interface, draw 
		
		playNow image at interface, draw exitGame image at interface
		
		Exception: None\\
		\\	
		update(arg0, arg1, delta):
		
		Input: GameContainer, StateBasedGame, init
		
		Transition: set default map, set mapToLoad = new TDMap("res/Try1.TDMap"), set new
		
		GameApplicationFrame(mapToLoad).
		
		Exception: None\\
		\\
		getID():
		
		Input: none
		
		Transition: none
		
		Output: return 0
		
		Exception: None\\
		\\	
\section{MIS of MenuApplicationFrame Module}
	\subsection{Interface Syntax}
	    \subsubsection{Exported Constants}
	    PIXELWIDTH=460\\
	    PIXELHEIGHT=200\\
	    APPNAME = "Main Menu"\\
	    TIMEOUT = 30\\
	    bPlay = Play a game\\
	    bCreateMap = Create a map\\
	    bQuit = Quit\\
	    bLoadMap = Load a map\\
	    bDefault = Default\\
	    lblMapToLoad = MAP: Default\\
		\subsubsection{Exported Access Programs}
		
	\begin{tabular}[pos]{|c|c|c|c|}
	\hline
	\textbf{Name}& \textbf{In} & \textbf{Out} & \textbf{Exceptions} \\ 
	\hline
	MenuApplicationFrame & - & - & - \\ \hline
	actionPerformed & ActionEvent& - & - \\ \hline
	init & - & - & - \\ \hline
	setMapName & string & - & - \\ \hline
	
	
					
	\end{tabular}		
		
	\subsection{Interface Semantics}
		\subsubsection{State Variables}
		mapToLoad: TDMap\\
	    fc: JFileChooser\\
	    mainPanel: JPanel\\
	    bPlay: JButton\\
	    bCreateMap: JButton\\
	    bQuit: JButton\\
	    bLoadMap: JButton\\
	    bDefault: JButton\\
	    lblMapToLoad: JLabel\\
	    e: ActionEvent\\
	    
		\subsubsection{Environmental Variables}
		None
		\subsubsection{Assumptions}
        None

		\subsubsection{Access Program Semantics}
		MenuApplicationFrame():
		
		Input: none
		
		Transition: set default map, set all button application
		
		Exception: None\\
		\\
		actionPerformed(e):
		
		Input: ActionEvent
		
		Transition: set every button to correspond perfermed action
		
		Exception: None\\
		\\	
		init():
		
		Input: none
		
		Transition: set panel properties, set mainPanel.setBackground(Color.BLACK), set 
		
		mainPanel.add(bCreateMap), set the Frame properties, set the x button as the 
		
		default close operation
		
		Exception: None\\
		\\
		setMapName(name):
		
		Input: string
		
		Transition: add name to map
		
		Exception: None\\
		\\
\section{MIS of SetupClass Module}
	\subsection{Interface Syntax}
		\subsubsection{Exported Access Programs}
		
	\begin{tabular}[pos]{|c|c|c|c|}
	\hline
	\textbf{Name}& \textbf{In} & \textbf{Out} & \textbf{Exceptions} \\ 
	\hline
	etupClass & string & - & - \\ \hline
	initStatesList & GameContainer& - & - \\ \hline

					
	\end{tabular}		
		
	\subsection{Interface Semantics}
		\subsubsection{State Variables}
		title: string\\
		container: GameContainer\\ 
	    
		\subsubsection{Environmental Variables}
		None
		\subsubsection{Assumptions}
        None

		\subsubsection{Access Program Semantics}
		SetupClass():
		
		Input: string
		
		Transition: set up title 
		
		Exception: None\\
		\\
		initStatesList(container):
		
		Input: GameContainer
		
		Transition: initial mainmenu and gamestate
		
		Exception: None\\
		\\	
		
\end{document}
