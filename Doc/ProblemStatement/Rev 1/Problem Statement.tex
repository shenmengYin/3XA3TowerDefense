\documentclass[12pt]{article}

\oddsidemargin 0mm
\evensidemargin 0mm
\textwidth 160mm
\textheight 200mm
\renewcommand\baselinestretch{1.0}

\pagestyle {plain}
\pagenumbering{arabic}

\newcounter{stepnum}

\title{Problem Statement}
\author{SFWR 3XA3}
\begin{document}
   \maketitle
   \noindent
   Group 30
   \newline
   Team Name : Team 30
   \newline
   Student Name : Alan Yin    MacId : yins1
   \newline
   Student Name : Junni Pan    MacId : panj10
   \newline
   Student Name : Huajie Zhu    MacId : zhuh5
   \section *{What problem are you trying to solve?}
   Many people play video games as a way of entertaining during leisure time, and there has been a variety of video games for different platforms on the market. Over these year, there is an increase in popularity of casual games, because they are easy to learn and they are not time consuming. As game developers, we are trying to create a simple tower defense game which can be played during lunch break or daily commute. We are working on an existing project, to add more feature and make the user interface friendlier.
   \newline
   \section *{Why is this an important problem?}
   There are a lot of existing tower defense games on the market, however most of them are not perfect for an instant entertainment. For example, many games ask the player to either keep playing the game for a long time or spend money to unlock new features, and some of the stages are too hard for casual gamers. Our recreation is aimed to make refinement on map creator and high-score feature, which makes the game unique. These additional features allow the players to create their own stages and seeing the score they reached after each gameplay. It can provide a feeling of accomplishment to the players, and make the game more enjoyable for entertaining purpose. 
   \newline
   \section *{What is the context of the problem you are solving?}
   The stakeholders of this game are potential players since we are providing them with an instant entertainment solution. Stakeholders also include ourselves as current developers and any future developers if exist, because we are practising the entire software developing process through this project, and future developers can make use of our documentations to better understand the open source project. The software is written in Java. It can run under Windows, Linux or MacOS, and on any device that support Java applications. 

\end{document}